\section{Заключение}

В первой части данной работы было рассмотрено численное моделирование динамических процессов, происходящих в ледовом острове при бурении грунта и статической нагрузке острова. Анализ волновых картин, возникающих при моделировании бурения, показал, что ледовый остров проявляет свойства резонатора волн упругости. Это свидетельствует о наличии риска резонансного разрушения льда при наличии периодических источников возмущений вблизи острова. Также был проведён анализ распределений напряжений фон Мизеса в ледовом острове при 100-тонной статической нагрузке с 5-метровым основанием. Данная нагрузка оказалась недостаточной для разрушения льда. Наиболее нагруженными частями острова были цилиндрическая область непосредственно под основанием статической нагрузки, а также конусообразные области радиусом около 20 метров с вершиной в центре острова и вертикальной осью.

Во второй части данной работы была теоретически показана возможность применения поглощающих граничных условий типа Berenger PML и split-field PML совместно с сеточно-харак\-теристическим методом. Был поставлен и проведён численный эксперимент для сравнения эффективности сеточно-характеристических и конечно-разностных реализаций этих граничных условий, а также поглощающего граничного условия Mur. Анализ результатов показал превосходство сеточно-харак\-теристических реализаций PML методов над конечно-разностными. Также был предложен метод комбинирования граничных условий типов PML и Mur для улучшения качества поглощения. 

Полученные результаты могут быть использованы для решения задач об использовании искусственных ледовых островов и улучшения точности численного моделирования распространения упругих волн в задачах прикладной вычислительной геофизики.