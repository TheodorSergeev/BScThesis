\section{Введение}

Арктический регион имеет огромные запасы полезных ископаемых. Например, суммарный объём одних только газовых месторождений на шельфе северных морей достигает 2.7 трлн тонн. Поиск, разработка и эксплуатация новых месторождений перспективны, но требуют решения новых вычислительных и инженерных задач для обеспечения эффективной и безопасной работы в Арктике \cite{petrov_arctic}.

В настоящее время особенный интерес для нефтегазовой индустрии представляет добыча полезных ископаемых на Арктическом шельфе с использованием искусственных ледовых островов. 

Ледовые острова обладают рядом преимуществ перед традиционными бетонными и металлическими нефтегазовыми платформами. Во-первых, основной строительный материал, лёд, в Арктике доступен и дёшев. Во-вторых, при использовании льда платформа является абсолютно экологически чистой. В третьих, в летний период лёд тает сам по себе, тем самым избавляя от необходимости проведения полного демонтажа несущих конструкций при завершении работы платформы. Это особенно важно для упрощения и удешевления разведочного бурения. Описанные преимущества делают ледовые острова отличным инструментом для проведения разведочного бурения в мелководных районах Северных морей.

При использовании ледовых островов возникает и ряд проблем. Важнейшей является обеспечение безопасности персонала и установок, находящихся на  поверхности острова. Устойчивости и целостности льда угрожают как механические, так и термические воздействия. К механическим воздействиям относятся сейсмическая активность \cite{ice_during_earthquake}, столкновение с айсбергами и ледовыми полями \cite{iceberg_crash, iceberg_crash2}, бурение и статическая нагрузка \cite{epifanov_crash}. К термическим --- воздействие солнечной радиации и тёплых течений  \cite{canadian_arctic, petrov_arctic}. Другой проблемой, возникающей при эксплуатации ледовых островов, оказывается значительное влияние льда на сейсморазведку \cite{stogniy_ice_influence}. Отражение упругих волн от поверхностей острова усложняет сейсмограммы, затрудняя их анализ и утяжеляя поиск полезных ископаемых.

Для расчёта механических воздействий, оказываемых на ледовый остров, требуется численное моделирование распространения упругих волн во льду и геологических средах. Для этих целей хорошо зарекомендовал себя  сеточно-характеристический метод. С его помощью можно решать задачи как на прямоугольных, так и на тетраэдральных сетках. Это позволяет применять его для моделирования неоднородных и трещиноватых сред. К его достоинствам также относится возможность постановки корректных граничных и контактных условий. Кроме того, сеточно-характеристический метод эффективно распараллеливается, позволяя производить объёмные расчёты на многопроцессорных вычислительных системах.  \cite{petrov_arctic, zhdanov_gcm, biryukov_fractured_layers, favorskaya_thesis, grigoriev}

При моделировании распространения волн в геологических средах часто используются поглощающие граничные условия \cite{seismo_pml,arch_comp_sim}. Самым простым поглощающим условием , пожалуй, является граничное условие Mur \cite{arch_comp_sim}. Граничные условиями типа полностью согласованного слоя (PML) --- более сложные, но и более эффективные. Разработано множество вариантов граничных условий PML, в частности, Berenger PML \cite{berenger} и split-field PML \cite{split_field_pml}. Граничные условия Mur, Berenger PML и split-field PML используются, как правило, совместно с конечно-разностным методом. Интересно  изучить возможность их применения совместно с  сеточно-характеристическим методом.

Данная работа состоит из двух частей. В первой части рассмотрено компьютерное моделирование динамических процессов в ледовом острове, в том числе задача о бурении и статической нагрузке. Во второй части рассмотрены поглощающие граничные условия типа  Mur, Berenger PML и split-field PML для двумерной системы уравнений акустики, проведён анализ их эффективности при использовании конечно-разностного и сеточно-характеристического методов.