\thispagestyle{plain}
\begin{center}
    \LARGE
    \textbf{Аннотация}
\end{center}

Цель данной работы --- проведение компьютерного моделирования волновых процессов, происходящих при эксплуатации искусственных ледовых островов, а также исследование применения поглощающих граничных условий типа PML совместно с сеточно-характеристическим методом для задач вычислительной геофизики.

В рамках работы проведено моделирование распространения волн упругости в ледовом острове, воде и грунте при бурении; найдены распределения напряжений в ледовом острове при статической нагрузке и выявлены области острова, наиболее подверженные разрушению.

Также в данной работе была теоретически доказана возможность применения поглощающих граничных условий Bere\-nger PML и split-field PML совместно с сеточно-характеристическим методом для двумерной системы уравнений акустики. Для этих граничных условий проведён численный эксперимент по сравнению эффективности работы конечно-разностной и сеточно-харак\-теристической реализаций, показавший превосходство последних.

\newpage